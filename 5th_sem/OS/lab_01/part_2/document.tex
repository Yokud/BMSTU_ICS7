\documentclass[a4paper,12pt]{extreport}

\usepackage{cmap}
\usepackage[T2A]{fontenc}
\usepackage[utf8]{inputenc}
\usepackage[english,russian]{babel}

\usepackage{amsmath}
\usepackage{float}

\usepackage{geometry}
\geometry{left=30mm}
\geometry{right=15mm}
\geometry{top=20mm}
\geometry{bottom=20mm}

\usepackage{titlesec}
\titleformat{\section}
{\normalsize\bfseries}
{\thesection}
{1em}{}
\titlespacing*{\chapter}{0pt}{-30pt}{8pt}
\titlespacing*{\section}{\parindent}{*4}{*4}
\titlespacing*{\subsection}{\parindent}{*4}{*4}

\usepackage{setspace}
\onehalfspacing

\frenchspacing
\usepackage{indentfirst}

\usepackage{titlesec}
\titleformat{\chapter}{\LARGE\bfseries}{\thechapter}{18pt}{\LARGE\bfseries}
\titleformat{\section}{\Large\bfseries}{\thesection}{18pt}{\Large\bfseries}

\usepackage{xcolor}


\usepackage{pgfplots}
\usetikzlibrary{datavisualization}
\usetikzlibrary{datavisualization.formats.functions}

\usepackage{graphicx}
\newcommand{\img}[3] {
	\begin{figure}[h]
		\center{\includegraphics[height=#1]{assets/img/#2}}
		\caption{#3}
		\label{img:#2}
	\end{figure}
}

\newcommand{\imgw}[3] {
	\begin{figure}[h]
		\center{\includegraphics[width=#1]{assets/img/#2}}
		\caption{#3}
		\label{img:#2}
	\end{figure}
}

\usepackage[justification=centering]{caption}
\usepackage[unicode,pdftex]{hyperref}
\hypersetup{hidelinks}
\newcommand{\code}[1]{\texttt{#1}}
\usepackage{icomma}
\usepackage{csvsimple}
\usepackage{svg}

\newcommand\Tstrut{\rule{0pt}{2.6ex}}       % "top" strut
\newcommand\Bstrut{\rule[-0.9ex]{0pt}{0pt}} % "bottom" strut
\newcommand{\TBstrut}{\Tstrut\Bstrut} % top&bottom struts

\begin{document}
\newgeometry{pdftex, left=2cm, right=2cm, top=2.5cm, bottom=2.5cm}
\fontsize{12pt}{12pt}\selectfont
\noindent \begin{minipage}{0.15\textwidth}
	\includegraphics[width=\linewidth]{b_logo.jpg}
\end{minipage}
\noindent\begin{minipage}{0.9\textwidth}\centering
	\textbf{Министерство науки и высшего образования Российской Федерации}\\
	\textbf{Федеральное государственное бюджетное образовательное учреждение высшего образования}\\
	\textbf{«Московский государственный технический университет имени Н.Э.~Баумана}\\
	\textbf{(национальный исследовательский университет)»}\\
	\textbf{(МГТУ им. Н.Э.~Баумана)}
\end{minipage}

\noindent\rule{18cm}{3pt}
\newline\newline
\noindent ФАКУЛЬТЕТ $\underline{\text{«Информатика и системы управления»}}$ \newline\newline
\noindent КАФЕДРА $\underline{\text{«Программное обеспечение ЭВМ и информационные технологии»}}$\newline\newline\newline\newline\newline\newline\newline


\begin{center}
	\Large\textbf{Отчет по лабораторной работе №1 (часть 2) по дисциплине <<Операционные системы>>}
\end{center}

""\newline\newline

\noindent\textbf{Тема} $\underline{\text{Прерывания таймера в Windows и UNIX~~~~~~~}}$\newline\newline
\noindent\textbf{Студент} $\underline{\text{Малышев И. А.~~~~~~~~~~~~~~~~~~~~~~~~~~~~~~~~~~~~}}$\newline\newline
\noindent\textbf{Группа} $\underline{\text{ИУ7-51Б~~~~~~~~~~~~~~~~~~~~~~~~~~~~~~~~~~~~~~~~~~~~~~~}}$\newline\newline
\noindent\textbf{Преподаватель} $\underline{\text{Рязанова Н. Ю.~~~~~~~~~~~~~~~~~~~~~~~~~}}$\newline

\begin{center}
	\vfill
	Москва~---~\the\year
	~г.
\end{center}
\restoregeometry

\newpage

\chapter{Функции обработчика прерывания от системного таймера}

\section{UNIX-системы}

\noindent Функции обработчика прерывания от системного таймера \textbf{по тику}:
\begin{itemize}
	\item инкремент счётчика тиков аппаратного таймера;
	\item декремент кванта текущего потока;
	\item обновление статистики использования процессора текущим процессом - инкремент поля \textbf{p\_cpu} дескриптора текущего процесса до максимального значения 127;
	\item инкремент часов и других таймеров системы;
	\item декремент счётчика времени до запуска обработчика отложенных вызовов, при достижении счётчиком нулевого значения - установка флага для обработчика отложенных вызовов.
\end{itemize}

""\newline
Функции обработчика прерывания от системного таймера \textbf{по главному тику}:
\begin{itemize}
	\item выход из состояния прерываемого сна в нужные моменты системных вызовов \textbf{swapper} и \textbf{pagedaemon};
	\item запись функций планировщика-диспетчера в таблицу отложенных вызовов, таких как пересчёт приоритетов;
	\item декремент счётчиков времени, значение которых показывает оставшееся время до отправления одного из сигналов тревоги:
	\begin{itemize}
		\item \textbf{SIGALRM} -- сигнал будильника реального времени, который отправляется ядром процессу по истечении заданного промежутка реального времени;
		\item \textbf{SIGPROF} -- сигнал будильника профиля процесса, который измеряет время работы процесса;
		\item \textbf{SIGVTALRM} -- сигнал будильника виртуального времени, который измеряет время работы процесса в режиме задачи.
	\end{itemize} 
\end{itemize}

""\newline
Функции обработчика прерывания от системного таймера \textbf{по кванту}:
\begin{itemize}
	\item отправка текущему процессу сигнала \textbf{SIGXCPU}, если тот превысил выделенную ему квоту использования процессора.
\end{itemize}

\section{Windows-системы}

\noindent Функции обработчика прерывания от системного таймера \textbf{по тику}:
\begin{itemize}
	\item инкремент счётчика системного времени;
	\item декремент остатка кванта текущего потока;
	\item декремент счётчиков времени отложенных задач;
	\item если активен механизм профилирования ядра, происходит постановка объекта \textbf{DPC} в \textbf{DPC}-очередь: обработчик ловушки профилирования регистрирует адрес команды, выполнявшейся на момент прерывания.
\end{itemize}

""\newline
Функции обработчика прерывания от системного таймера \textbf{по главному тику}:
\begin{itemize}
	\item сброс объекта <<событие>>, на котором диспетчер настройки баланса ожидает.
\end{itemize}

""\newline
Функции обработчика прерывания от системного таймера \textbf{по кванту}:
\begin{itemize}
	\item добавление объекта \textbf{DPC} в \textbf{DPC}-очередь.
\end{itemize}
\newpage

\chapter{Пересчёт динамических приоритетов}

Динамические приоритеты могут быть только у пользовательских процессов.

\section{UNIX-системы}

В современных системах \textbf{UNIX} ядро является вытесняющим, то есть процесс в режиме ядра может быть вытеснен более приоритетным процессом, находящимся так же в режиме ядра. Это было сделано для того, чтобы система могла обслуживать одновременно процессы интерактивных и фоновых приложений.

Согласно приоритетам процессов и принципу вытесняющего циклического планирования формируется очередь готовых к выполнению процессов. В первую очередь выполняются процессы с большим приоритетом. В случае, если процесс, имеющий более высокий приоритет, поступает в очередь готовых к выполнению процессов, планировщик вытесняет текущий процесс, даже если он не отработал свой временной квант, и предоставляет ресурс процессу с большим приоритетом.

Приоритет пользовательского процесса может иметь значение от 50 до 127. Их приоритет может изменяться во времени в зависимости от следующих факторов:

\begin{itemize}
	\item фактор любезности;
	\item последней измеренной величины использования процессора.
\end{itemize}

Фактор любезности -- это целое число в диапазоне от 0 до 39 (по умолчанию 20). Чем меньше значение фактора любезности процесса, тем выше приоритет процесса. Фактор любезности процесса может быть уменьшен только суперпользователем с помощью системного вызова \textbf{nice}. Фоновым процессам задаются более высокие значения фактора любезности.

Дескриптор процесса \textbf{proc} содержит следующие поля, которые относятся к приоритету процесса:

\begin{itemize}
	\item \textbf{p\_pri} -- текущий приоритет планирования;
	\item \textbf{p\_usrpri} -- приоритет процесса в режиме задачи;
	\item \textbf{p\_cpu} -- результат последнего измерения использования процессора;
	\item \textbf{p\_nice} -- фактор любезности, устанавливаемый пользователем.
\end{itemize}

Когда процесс находится в режиме задачи, значения \textbf{p\_pri} и \textbf{p\_usrpri} равны. Значение текущего приоритета \textbf{p\_pri} может быть повышено планировщиком для выполнения процесса в режиме ядра, а \textbf{p\_usrpri} будет использоваться для хранения приоритета, который будет назначен при переходе процесса из режима ядра в режим задачи.

Ядро системы связывает приоритет сна с событием или ожидаемым ресурсом, из-за которого процесс может быть заблокирован. В тот момент, когда процесс просыпается, после того как был заблокирован в системном вызове, ядро устанавливает приоритет сна в поле \textbf{p\_pri} -- это значение приоритета в диапазоне от 0 до 49, зависящее от события или ресурса, по которому произошла блокировка. В таблице \ref{tab:bsd} приведены значения приоритетов сна для систем \textbf{4.3BSD}.


\begin{table}[H]
	\caption{Таблица приоритетов сна в системе \textbf{4.3BSD}}
	\label{tab:bsd}
	\begin{center}
		\begin{tabular}{ |c|c|c|  }
			\hline
			\textbf{Приоритет} & \textbf{Значение} & \textbf{Описание} \\
			\hline
			\texttt{PSWP} & 0 & Свопинг \\
			\hline
			\texttt{PSWP + 1} & 1 & Страничный демон \\
			\hline
			\texttt{PSWP + 1/2/4} & 1/2/4 & Другие действия по обработке памяти \\
			\hline
			\texttt{PINOD} & 10 & Ожидание освобождения inode \\
			\hline
			\texttt{PRIBIO} & 20 & Ожидание дискового ввода-вывода \\
			\hline
			\texttt{PRIBIO + 1} & 21 & Ожидание освобождения буфера \\
			\hline
			\texttt{PZERO} & 25 & Базовый приоритет \\
			\hline
			\texttt{TTIPRI} & 28 & Ожидание ввода с терминала \\
			\hline
			\texttt{TTOPRI} & 29 & Ожидание вывода с терминала \\
			\hline 
			\texttt{PWAIT} & 30 & Ожидание завершения процесса потомка \\
			\hline
			\texttt{PLOCK} & 35 & Консультативное ожидание блокированного ресурса \\
			\hline
			\texttt{PSLEP} & 40 & Ожидание сигнала \\
			\hline
		\end{tabular}
	\end{center}
\end{table}

При создании процесса поле \textbf{p\_cpu} инициализируется нулём. На каждом тике обработчик таймера увеличивает это поле для текущего процесса на единицу, до максимального значения, которое равно 127. Каждую секунду обработчик прерывания инициализирует отложенный вызов процедуры \textbf{schedcpy()}, которая уменьшает значение \textbf{p\_cpu} каждого процесса исходя из фактора <<полураспада>>. В системе \textbf{4.3BSD} фактор полураспада рассчитывается по формуле \eqref{for:bsd}: 

\begin{equation}
	\label{for:bsd}
	decay = \frac{2 \cdot load\_average}{2 \cdot load\_average + 1}
\end{equation}
где \textbf{load\_average} -- среднее количество процессов, находящихся в состоянии готовности к выполнению (за последнюю секунду).

Приоритеты для режима задачи всех процессов в процедуре \textbf{schedcpy()} пересчитываются по формуле \eqref{for:sched}:

\begin{equation}
	\label{for:sched}
	p\_usrpri = PUSER + \frac{p\_cpu}{2} + 2 \cdot p\_nice
\end{equation}
где \textbf{PUSER} -- базовый приоритет в режиме задачи, который равен 50.

Если процесс в последний раз использовал большое количество процессорного времени, его \textbf{p\_cpu} будет увеличен. Это приведёт к росту значения \textbf{p\_usrpri}, что приведет к понижению приоритета. Чем дольше процесс простаивает в очереди на исполнение, тем больше фактор полураспада уменьшает его \textbf{p\_cpu}, что приводит к повышению его приоритета. Данная схема предотвращает зависание низкоприоритетных процессов по вине операционной системы. Применение такой схемы предпочтительнее процессам, которые осуществляют много операций ввода-вывода, в противоположность процессам, производящим много вычислений.


\section{Windows-системы}

В системах Windows реализовано вытесняющее планирование на основе уровней приоритета, при котором выполняется готовый поток с наивысшим приоритетом. При создании процесса для него назначается базовый приоритет. Относительно базового приоритета процесса потоку назначается приоритет.

Планирование осуществляется следующим образом: поток с более низким приоритетом вытесняется потоком с более высоким приоритетом, в тот момент, когда этот поток становится готовым к выполнению. По истечению кванта времени текущего потока, ресурс передается самому приоритетному потоку в очереди готовых на выполнение.

Раз в секунду диспетчер настройки баланса сканирует очередь готовых потоков, и, в случае, если обнаружены потоки, ожидающие выполнения более 4 секунд, диспетчер настройки баланса повышает их приоритет до 15. Как только квант истекает, приоритет потока снижается до базового приоритета. В случае, если поток не был завершен за квант времени или был вытеснен потоком с более высоким приоритетом, то после снижения приоритета поток возвращается в очередь готовых потоков.

Для того чтобы минимизировать расход процессорного времени, диспетчер настройки баланса сканирует только 16 готовых потоков. Диспетчер повышает приоритет не более чем у 10 потоков за один проход. Если он обнаруживает 10 потоков, приоритет которых следует повысить, он прекращает сканирование. При следующем проходе сканирование возобновляется с того места, где оно было прервано. Наличие 10 потоков, приоритет которых нужно повысить, говорит о высокой загруженности системы.

Windows использует 32 уровня приоритета, которые описываются целыми числами от 0 до 31. Приоритеты от 16 до 31 -- уровни реального времени, от 1 до 15 -- динамические уровни. Системный уровень 0 зарезервирован для потока обнуления страниц.

Уровни приоритета потоков назначаются с двух позиций: \textbf{Windows API} и ядра операционной системы.\textbf{Windows API} сортирует процессы по классам приоритета, которые были назначены при их создании:

\begin{itemize}
	\item реального времени (real-time);
	\item высокий (high);
	\item выше обычного (above normal);
	\item обычный (normal);
	\item ниже обычного (below normal).
	\item простой (idle).
\end{itemize}

""\newline
Затем назначается относительный приоритет потоков в рамках процессов:

\begin{itemize} 
	\item критичный по времени (time critical, 15);
	\item наивысший (highest, 2);
	\item выше обычного (above normal, 1);
	\item обычный (normal, 0);
	\item ниже обычного (below normal, -1);
	\item низший (lowest, -2);
	\item простой (idle, -15).
\end{itemize} 

""\newline
Исходный базовый приоритет потока наследуется от базового приоритета процесса. Процесс по умолчанию наследует свой базовый приоритет у того процесса, который его создал. 

Таким образом, в \textbf{Windows API} каждый поток имеет базовый приоритет, являю­щийся функцией класса приоритета процесса и его относительного приоритета процесса. В  ядре класс приоритета процесса преобразуется в базовый приоритет. В таблице \ref{tbl:priority} приведено соответствие между приоритетами \textbf{Windows API} и ядра системы приоритета.

\begin{table}[H]
	\caption{Соответствие между приоритетами \textbf{Windows API} и ядра Windows}
	\begin{center}
		\begin{tabular}{|l|p{45pt}|p{45pt}|p{45pt}|p{45pt}|p{45pt}|p{45pt}|}
			\hline
			{} & \textbf{real-time} & \textbf{high} & \textbf{above normal} & \textbf{normal} & \textbf{below normal} & \textbf{idle}\\
			\hline
			\textbf{time critical} & 31 & 15 & 15 & 15 & 15 & 15 \\
			\hline
			\textbf{highest} & 26 & 15 & 12 & 10 & 8 & 6 \\
			\hline
			\textbf{above normal} & 25 & 14 & 11 & 9 & 7 & 5 \\
			\hline
			\textbf{normal} & 24 & 13 & 10 & 8 & 6 & 4 \\
			\hline
			\textbf{below normal} & 23 & 12 & 9 & 7 & 5 & 3 \\
			\hline
			\textbf{lowest} & 22 & 11 & 8 & 6 & 4 & 2 \\
			\hline
			\textbf{idle} & 16 & 1 & 1 & 1 & 1 & 1 \\
			\hline
		\end{tabular}
	\end{center}
	\label{tbl:priority}
\end{table}


Текущий приоритет потока в динамическом диапазоне (от 1 до 15) может быть изменён планировщиком вследствие причин:

\begin{itemize}
	\item повышение приоритета после завершения операций ввода-вывода;
	\item повышение приоритета владельца блокировки;
	\item повышение приоритета вследствие ввода из пользовательского интерфейса;
	\item повышение приоритета вследствие длительного ожидания ресурса исполняющей системы;
	\item повышение приоритета вследствие ожидания объекта ядра;
	\item повышение приоритета в случае, когда готовый к выполнению поток не был запущен в течение длительного времени;
	\item повышение приоритета проигрывания мультимедиа службой планировщика \textbf{MMCSS}.
\end{itemize}

Текущий приоритет потока в динамическом диапазоне может быть понижен до базового путем вычитания всех его повышений. В таблице \ref{tab:io} приведены рекомендуемые значения повышения приоритета для устройств ввода-вывода.

\begin{table}[H]
	\caption{Рекомендуемые значения повышения приоритета.}
	\begin{center}
		\begin{tabular}{|p{100mm}|l|}
			\hline
			\textbf{Устройство} & \textbf{Приращение} \\
			\hline
			Диск, CD-ROM, параллельный порт, видео & 1 \\
			\hline
			Сеть, почтовый ящик, именованный канал, последовательный порт & 2 \\
			\hline
			Клавиатура, мышь & 6 \\
			\hline
			Звуковая плата & 8 \\
			\hline
		\end{tabular}
	\end{center}
	\label{tab:io}
\end{table}

\subsection{MMCSS}

Потоки, на которых выполняются различные мультимедийные приложения, должны выполняться с минимальными задержками. В Windows эта задача решается путем повышения приоритетов таких потоков драйвером \textbf{MMCSS} -- MultiMedia Class Scheduler Service. Приложения, которые реализуют воспроизведение мультимедиа, указывают драйверу \textbf{MMCSS} задачу из списка:

\begin{itemize}
	\item звук;
	\item игры;
	\item захват изображения с экрана;
	\item воспроизведение медиаконтента;
	\item профессиональное аудио (Pro Audio);
	\item диспетчер окон;
	\item задачи администратора многоэкранного режима.
\end{itemize}

Одно из наиболее важных свойств для планирования потоков -- категория планирования -- первичный фактор, определяющий приоритет потоков, зарегистрированных в \textbf{MMCSS}. Различные категории планирования представленны в таблице \ref{tab:plan}.

\begin{table}[H]
	\caption{Категории планирования.}
	\begin{center}
		\begin{tabular}{|p{40mm}|p{30mm}|p{80mm}|}
			\hline
			\textbf{Категория} & \textbf{Приоритет} & \textbf{Описание} \\
			\hline
			High (Высокая) & 23-26 & Потоки профессионального аудио (Pro Audio), запущенные с приоритетом выше, чем у других потоков на системе, за исключением критических системных потоков \\
			\hline
			Medium (Средняя) & 16-22 & Потоки, являющиеся частью приложений первого плана, например, Windows Media Player \\
			\hline
			Low (Низкая) & 8-15 & Все остальные потоки, не являющиеся частью предыдущих категорий \\
			\hline
			Exhausted (Исчерпавших потоков) & 1-7 & Потоки, исчерпавшие свою долю времени центрального процессора, выполнение которых продолжиться, только если не будут готовы к выполнению другие потоки с более высоким уровнем приоритета \\
			\hline
		\end{tabular}
	\end{center}
	\label{tab:plan}
\end{table}

Функции \textbf{MMCSS} временно повышают приоритет потоков, зарегистрированных с \textbf{MMCS} до уровня, соответствующего их категориям планирования. Далее, их приоритет снижается до уровня, соответствующего категории \textbf{Exhausted}, для того чтобы другие потоки могли получить ресурс.

\chapter{Вывод}

Обработчик прерываний от системного таймера в системах UNIX и Windows имеют схожие функции:

\begin{itemize}
	\item инкремент счётчика системного времени;
	\item декремент кванта;
	\item создание отложенных действий, относящихся к работе планировщика.
\end{itemize}

Это объясняется тем, что обе системы являются системами разделения времени с динамическими приоритетами и вытеснением. При этом они различаются подходом к планированию и способами пересчёта приоритетов как процессов, так и потоков. Пересчёт динамических приоритетов пользовательских процессов выполняется для исключения возможности бесконечного откладывания процессов.

\end{document}