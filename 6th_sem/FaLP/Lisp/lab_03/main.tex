\documentclass[12pt]{report}
\usepackage[utf8]{inputenc}
\usepackage[english, russian]{babel}
\usepackage{listings}
\usepackage{graphicx}
\usepackage{float}
\graphicspath{{imgs/}}
\usepackage{amsmath,amsfonts,amssymb,amsthm,mathtools} 
\usepackage{pgfplots}
\usepackage{filecontents}
\usepackage{indentfirst}
\usepackage{eucal}
\usepackage{enumitem}
\frenchspacing

\usepackage{indentfirst} % Красная строка

\usetikzlibrary{datavisualization}
\usetikzlibrary{datavisualization.formats.functions}

\usepackage{amsmath}
\usepackage{fixltx2e}
\usepackage{caption}


\definecolor{bluekeywords}{rgb}{0,0,1}
\definecolor{greencomments}{rgb}{0,0.5,0}
\definecolor{redstrings}{rgb}{0.64,0.08,0.08}
\definecolor{xmlcomments}{rgb}{0.5,0.5,0.5}
\definecolor{types}{rgb}{0.17,0.57,0.68}

\usepackage{listings}
\lstset{language=[Sharp]C,
	captionpos=t,
	numbers=left, %Nummerierung
	numberstyle=\small, % kleine Zeilennummern
	frame=single, % Oberhalb und unterhalb des Listings ist eine Linie
	stepnumber=1,                   
	numbersep=5pt,                
	showspaces=false,
	tabsize=2,
	showtabs=false,
	breaklines=true,
	showstringspaces=false,
	breakatwhitespace=true,
	escapeinside={(*@}{@*)},
	commentstyle=\color{greencomments},
	morekeywords={partial, var, value, get, set},
	keywordstyle=\color{bluekeywords},
	stringstyle=\color{redstrings},
	basicstyle=\ttfamily\small,
}

\usepackage[left=2cm,right=2cm, top=2cm,bottom=2cm,bindingoffset=0cm]{geometry}
% Для измененных титулов глав:
\usepackage{titlesec, blindtext, color} % подключаем нужные пакеты
\definecolor{gray75}{gray}{0.75} % определяем цвет
\newcommand{\hsp}{\hspace{20pt}} % длина линии в 20pt
% titleformat определяет стиль
\titleformat{\chapter}[hang]{\Huge\bfseries}{\thechapter\hsp\textcolor{gray75}{|}\hsp}{0pt}{\Huge\bfseries}

\usepackage{array}
\newcommand{\head}[2]{\multicolumn{1}{>{\centering\arraybackslash}p{#1}}{#2}}

% plot
\usepackage{pgfplots}
\usepackage{filecontents}
\usetikzlibrary{datavisualization}
\usetikzlibrary{datavisualization.formats.functions}

\begin{document}
%\def\chaptername{} % убирает "Глава"
\thispagestyle{empty}
\begin{titlepage}
	\noindent \begin{minipage}{0.15\textwidth}
		\includegraphics[width=\linewidth]{b_logo}
	\end{minipage}
	\noindent\begin{minipage}{0.9\textwidth}\centering
		\textbf{Министерство науки и высшего образования Российской Федерации}\\
		\textbf{Федеральное государственное бюджетное образовательное учреждение высшего образования}\\
		\textbf{~~~«Московский государственный технический университет имени Н.Э.~Баумана}\\
		\textbf{(национальный исследовательский университет)»}\\
		\textbf{(МГТУ им. Н.Э.~Баумана)}
	\end{minipage}
	
	\noindent\rule{18cm}{3pt}
	\newline\newline
	\noindent ФАКУЛЬТЕТ $\underline{\text{«Информатика и системы управления»}}$ \newline\newline
	\noindent КАФЕДРА $\underline{\text{«Программное обеспечение ЭВМ и информационные технологии»}}$\newline\newline\newline\newline\newline\newline\newline\newline\newline\newline\newline
	
	
	\begin{center}
		\noindent\begin{minipage}{1.3\textwidth}\centering
			\Large\textbf{  Отчет по лабораторной работе №3}\newline
			\textbf{по дисциплине \newline "Функциональное и логическое программирование"}\newline\newline
		\end{minipage}
	\end{center}
	
	\noindent\textbf{Тема} $\underline{\text{Работа интерпретатора Lisp}}$\newline\newline
	\noindent\textbf{Студент} $\underline{\text{Малышев И. А.}}$\newline\newline
	\noindent\textbf{Группа} $\underline{\text{ИУ7-61Б}}$\newline\newline
	\noindent\textbf{Оценка (баллы)} $\underline{\text{~~~~~~~~~~~~~~~~~~~~~~~~~~~}}$\newline\newline
	\noindent\textbf{Преподаватель: } $\underline{\text{Толпинская Н. Б.}}$\newline\newline\newline
	
	\begin{center}
		\vfill
		Москва~---~\the\year
		~г.
	\end{center}
\end{titlepage}


\setcounter{page}{2}
\chapter*{Теоретические вопросы}

\section*{1. Базис Lisp}

Базис языка представлен:
\begin{itemize}
	\item структурами и атомами;
	\item функциями;
\end{itemize}

Функция -- правило, по которому каждому значению одного или нескольких аргументов ставится в соответствие конкретное значение результата.

Функции, входящие в базис языка: atom, eq, cons, car, cdr, cond, quote, lambda, eval, apply, funcall.


\section*{2. Классификация функций}

\begin{itemize}
	\item "чистая" функция -- соответствует математической функции;
	\item специальные функции или формы -- могут принимать несколько аргументов;
	\item псевдофункции -- создают «эффект» -- отображение на экране процесса обработки данных и т.п.;
	\item функции с вариантами значения;
	\item функционалы -- принимают в качестве аргумента функцию или возвращают функцию;
	\item базисные функции (cons, car, cdr, atom, cond, eq, quote, eval, lambda, apply, funcall).
\end{itemize}

\section*{3. Способы создание функций}

Определение функций пользователя в Lisp-е возможно двумя способами:
\begin{itemize}
	\item с использованием Лямбда-нотаци (функции без имени): (lambda (<аргументы>) (<тело>));
	\item с использованием макро определения DEFUN: (defun <имя> (<аргументы>) (<тело>)).
\end{itemize}

\section*{4. Работа функций Cond, if, and/or}

Сигнатура функции \textbf{cond}:

\indent(cond (предикат-1 действие-1)) \newline
\indent(предикат-2 действие-2) \newline
\indent...\newline
\indent(предикат-n действие-n)\newline

\indent Работа функции \textbf{cond}: 

сначала просматриваются все предикаты в порядке следования, и если хоть один из них истинный, то cond возвращает результат, связанный с этим предикатом. Если ни один предикат не был истинным, то она вернет Nil. 

""\newline
Сигнатура функции \textbf{if}: 

(if условие выражение-1 выражение-2)\newline

\indent Работа функции \textbf{if}: 

если условие истинно (T), то выполняется выражение-1, иначе (Nil) – выражение-2\newline

""\newline
Сигнатура функции \textbf{and}: 

(and выражение-1 выражение-2 ... выражение-n)\newline

\indent Работа функции \textbf{and}: 

результат функции будет истинным, если все ее выражения истинны. В таком случае в качестве результата вернется значение выражения-n. В случае, если хотя бы одно выражение ложно (Nil), вычисление последующих выражений не производится и результатом функции является Nil.\newline

""\newline
Сигнатура функции \textbf{or}: 

(or выражение-1 выражение-2 ... выражение-n)\newline

Работа функции \textbf{or}: 

результат функции будет ложным (Nil), если все ее выражения ложны. В случае, если хотя бы одно выражение истинно, вычисление последующих выражений не производится и результатом функции является значения выражения, которое первым в списке аргументов дало в результате истину.\newline



""\newline
\chapter*{Практические задания}

\subsection*{1. Написать функцию, которая принимает целое число и возвращает первое четное число, не меньшее аргумента.}

\begin{lstlisting}[label=6xd, caption=Решение задания №1, language=lisp]
(defun first-even (x) (if (evenp x) x (+ x 1)))
\end{lstlisting}

\subsection*{2. Написать функцию, которая принимает число и возвращает число
	того же знака, но с модулем на 1 больше модуля аргумента.}

\begin{lstlisting}[label=6xd, caption=Решение задания №2, language=lisp]
(defun abs-plus-one (x) (if (> x 0) (+ x 1) (- x 1)))
\end{lstlisting}

\subsection*{3. Написать функцию, которая принимает два числа и возвращает
	список из этих чисел, расположенный по возрастанию.}

\begin{lstlisting}[label=6xd, caption=Решение задания №3, language=lisp]
(defun sorted-pair-list (fst snd) (if (> fst snd) (list fst snd) (list snd fst)))
\end{lstlisting}

\subsection*{4. Написать функцию, которая принимает три числа и возвращает Т только
	тогда, когда первое число расположено между вторым и третьим.}

\begin{lstlisting}[label=6xd, caption=Решение задания №4, language=lisp]
(defun between (a b c) (if (and (> a b) (< a c)) T Nil))
\end{lstlisting}

\subsection*{5. Каков результат вычисления следующих выражений?}

(and 'fee 'fie 'foe) -> foe

(or 'fee 'fie 'foe) -> fee

(or nil 'fie 'foe) -> fie

(and nil 'fie 'foe) -> NIL

(and (equal 'abc 'abc) 'yes) -> yes

(or (equal 'abc 'abc) 'yes) -> T

\subsection*{6. Написать предикат, который принимает два числа-аргумента и возвращает
	Т, если первое число не меньше второго.}

\begin{lstlisting}[label=6xd, caption=Решение задания №6, language=lisp]
(defun ge (a b) (if (>= a b) T Nil))
\end{lstlisting}

\subsection*{7. Какой из следующих двух вариантов предиката ошибочен и почему?}

(defun pred1 (x) (and (numberp x) (plusp x))) -- ok

(defun pred2 (x) (and (plusp x)(numberp x))) -- runtime error

""\newline
Второй вариант ошибочен, т.к. если в функцию будет передано не число и на него будет применена функция \textbf{plusp} (которая работает только с числовыми значениями), интерпретатор выдаст ошибку.

\subsection*{8. Решить задачу 4, используя для ее решения конструкции
	IF, COND, AND/OR.}

\begin{lstlisting}[label=6xd, caption=Решение задания №8, language=lisp]
(defun between-if (a b c) (if (and (> a b) (< a c)) T Nil))

(defun between-cond (a b c) (cond ((and (> a b) (< a c)) T) (T Nil)))

(defun between-and (a b c) (and (> a b) (< a c)))
\end{lstlisting}

\subsection*{9. Переписать функцию how-alike, приведенную в лекции и использующую COND, используя
	только конструкции IF, AND/OR.}

\begin{lstlisting}[label=6xd, caption=Решение задания №9, language=lisp]
(defun how-alike-cond (x y)
	(cond ((or (= x y) (equal x y)) `the_same)
		((and (oddp x) (oddp y)) `both_odd)
		((and (evenp x) (evenp y)) `both_even)
		(T `difference)))



(defun how-alike-if (x y)
	(if (if (= x y) (equal x y)) `the_same 
	(if (if (oddp x) (oddp y)) `both_odd 
	(if (if (evenp x) (evenp y)) `both_even 
	`difference))))

(defun how-alike-andor (x y)
	(or (and (or (= x y) (equal x y)) `the_same)
		(and (oddp x) (oddp y) `both_odd)
		(and (evenp x) (evenp y) `both_even) 
		`difference))
\end{lstlisting}

\end{document}