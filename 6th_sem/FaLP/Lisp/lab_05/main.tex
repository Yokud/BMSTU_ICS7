\documentclass[12pt]{report}
\usepackage[utf8]{inputenc}
\usepackage[english, russian]{babel}
\usepackage{listings}
\usepackage{graphicx}
\usepackage{float}
\graphicspath{{imgs/}}
\usepackage{amsmath,amsfonts,amssymb,amsthm,mathtools} 
\usepackage{pgfplots}
\usepackage{filecontents}
\usepackage{indentfirst}
\usepackage{eucal}
\usepackage{enumitem}
\frenchspacing

\usepackage{indentfirst} % Красная строка

\usetikzlibrary{datavisualization}
\usetikzlibrary{datavisualization.formats.functions}

\usepackage{amsmath}
\usepackage{fixltx2e}
\usepackage{caption}


\definecolor{bluekeywords}{rgb}{0,0,1}
\definecolor{greencomments}{rgb}{0,0.5,0}
\definecolor{redstrings}{rgb}{0.64,0.08,0.08}
\definecolor{xmlcomments}{rgb}{0.5,0.5,0.5}
\definecolor{types}{rgb}{0.17,0.57,0.68}

\usepackage{listings}
\lstset{language=[Sharp]C,
	captionpos=t,
	numbers=left, %Nummerierung
	numberstyle=\small, % kleine Zeilennummern
	frame=single, % Oberhalb und unterhalb des Listings ist eine Linie
	stepnumber=1,                   
	numbersep=5pt,                
	showspaces=false,
	tabsize=2,
	showtabs=false,
	breaklines=true,
	showstringspaces=false,
	breakatwhitespace=true,
	escapeinside={(*@}{@*)},
	commentstyle=\color{greencomments},
	morekeywords={partial, var, value, get, set},
	keywordstyle=\color{bluekeywords},
	stringstyle=\color{redstrings},
	basicstyle=\ttfamily\small,
}

\usepackage[left=2cm,right=2cm, top=2cm,bottom=2cm,bindingoffset=0cm]{geometry}
% Для измененных титулов глав:
\usepackage{titlesec, blindtext, color} % подключаем нужные пакеты
\definecolor{gray75}{gray}{0.75} % определяем цвет
\newcommand{\hsp}{\hspace{20pt}} % длина линии в 20pt
% titleformat определяет стиль
\titleformat{\chapter}[hang]{\Huge\bfseries}{\thechapter\hsp\textcolor{gray75}{|}\hsp}{0pt}{\Huge\bfseries}

\usepackage{array}
\newcommand{\head}[2]{\multicolumn{1}{>{\centering\arraybackslash}p{#1}}{#2}}

% plot
\usepackage{pgfplots}
\usepackage{filecontents}
\usetikzlibrary{datavisualization}
\usetikzlibrary{datavisualization.formats.functions}

\begin{document}
%\def\chaptername{} % убирает "Глава"
\thispagestyle{empty}
\begin{titlepage}
	\noindent \begin{minipage}{0.15\textwidth}
		\includegraphics[width=\linewidth]{b_logo}
	\end{minipage}
	\noindent\begin{minipage}{0.9\textwidth}\centering
		\textbf{Министерство науки и высшего образования Российской Федерации}\\
		\textbf{Федеральное государственное бюджетное образовательное учреждение высшего образования}\\
		\textbf{~~~«Московский государственный технический университет имени Н.Э.~Баумана}\\
		\textbf{(национальный исследовательский университет)»}\\
		\textbf{(МГТУ им. Н.Э.~Баумана)}
	\end{minipage}
	
	\noindent\rule{18cm}{3pt}
	\newline\newline
	\noindent ФАКУЛЬТЕТ $\underline{\text{«Информатика и системы управления»}}$ \newline\newline
	\noindent КАФЕДРА $\underline{\text{«Программное обеспечение ЭВМ и информационные технологии»}}$\newline\newline\newline\newline\newline\newline\newline\newline\newline\newline\newline
	
	
	\begin{center}
		\noindent\begin{minipage}{1.3\textwidth}\centering
			\Large\textbf{  Отчет по лабораторной работе №5}\newline
			\textbf{по дисциплине \newline "Функциональное и логическое программирование"}\newline\newline
		\end{minipage}
	\end{center}
	
	\noindent\textbf{Тема} $\underline{\text{Использование управляющих структур, работа со списками}}$\newline\newline
	\noindent\textbf{Студент} $\underline{\text{Малышев И. А.}}$\newline\newline
	\noindent\textbf{Группа} $\underline{\text{ИУ7-61Б}}$\newline\newline
	\noindent\textbf{Оценка (баллы)} $\underline{\text{~~~~~~~~~~~~~~~~~~~~~~~~~~~}}$\newline\newline
	\noindent\textbf{Преподаватель: } $\underline{\text{Толпинская Н. Б.}}$\newline\newline\newline
	
	\begin{center}
		\vfill
		Москва~---~\the\year
		~г.
	\end{center}
\end{titlepage}


\setcounter{page}{2}
\chapter*{Теоретические вопросы}

\section*{1. Структуроразрушающие и не разрушающие структуру списка функции}

\textbf{Не разрушающие структуру списка функции.} Данные функции не меняют сам объект-аргумент, а создают копию. К таким функциям относятся: 
\texttt{append, reverse, last, nth, nthcdr, length, remove, subst} и прочие.\\

\indent \textbf{Структуроразрушающие функции.} Данные функции меняют сам объект-аргумент, невозможно вернуться к исходному списку. Чаще всего такие функции начинаются с префикса \texttt{n-}. К такми функция относятся: \texttt{nreverse, nconc, nsubst} и прочие.\\



\section*{2. Отличие в работе функций cons, list, append, nconc и в их результате}

Функция \texttt{cons} --- чисто математическая, конструирует списковую ячейку, которая может вовсе и не быть списком. Является списком только в том случае, если вторым аргументом передан список.\\

Функция \texttt{list} --- форма, принимает произвольное количество аргументов и конструирует из них список. Результат --- всегда список. При нуле аргументов возвращает пустой список.\\

Функция \texttt{append} --- форма, принимает на вход произвольное количество аргументов и для всех аргументов, кроме последнего, создает копию, ссылая при этом последний элемент каждого списка-аргумента на первый элемент следующего по порядку списка-аргумента. Копирование для последнего не делается в целях эффективности.



""\newline
\chapter*{Практические задания}

\subsection*{1. Написать функцию, которая по своему списку-аргументу lst определяет
	является ли он палиндромом (то есть равны ли lst и (reverse lst)).}

\begin{lstlisting}[label=6xd, caption=Решение задания №1, language=lisp]
(defun is-palindrome (lst)
	(equal lst (reverse lst)))

\end{lstlisting}

\subsection*{2.Написать предикат set-equal, который возвращает t, если два его множества-аргумента содержат одни и те же элементы, порядок которых не имеет значения.}

\begin{lstlisting}[label=6xd, caption=Решение задания №2, language=lisp]
(defun set-equal (set1 set2)
	(if (= (length set1) (length set2))
		(and (subsetp set1 set2) (subsetp set2 set1))
		Nil))

\end{lstlisting}

\subsection*{3.  Напишите свои необходимые функции, которые обрабатывают таблицу из 4-х точечных
	пар:
	(страна . столица), и возвращают по стране - столицу, а по столице — страну .}

\begin{lstlisting}[label=6xd, caption=Решение задания №3, language=lisp]
(defun get-capital (table country)
	(cdr (assoc country table)))

(defun get-country (table capital)
	(car (rassoc capital table)))

\end{lstlisting}

\newpage
\subsection*{4.  Напишите функцию swap-first-last, которая переставляет в списке-аргументе первый и
	последний элементы.}

\begin{lstlisting}[label=6xd, caption=Решение задания №4, language=lisp]
(defun swap-first-last (lst)
	(let ((el1 (car lst)) (last-el (car (reverse lst))))
		(reverse (cons el1 (cdr (reverse (cons last-el (cdr lst))))))))

\end{lstlisting}

\subsection*{5. Напишите функцию swap-two-ellement, которая переставляет в списке- аргументе два
	указанных своими порядковыми номерами элемента в этом списке.}

\begin{lstlisting}[label=6xd, caption=Решение задания №5, language=lisp]
(defun swap-two-ellement (n1 n2 lst)
	(let ((len (length lst)) (lst-copy (copy-list lst)))
		(and (< n1 len) (< n2 len)
		(let ((el1 (nth n1 lst)) (el2 (nth n2 lst)))
			(setf (nth n1 lst-copy) el2)
			(setf (nth n2 lst-copy) el1)
			lst-copy))))
	
\end{lstlisting}

\subsection*{6. Напишите две функции, swap-to-left и swap-to-right, которые производят одну круговую
	перестановку в списке-аргументе влево и вправо, соответственно.}

\begin{lstlisting}[label=6xd, caption=Решение задания №6, language=lisp]
(defun swap-to-left (lst) 
	(append (cdr lst) (cons (car lst) Nil)))

(defun swap-to-right (lst)
	(cons
		(car (reverse lst))
		(reverse (cdr (reverse lst)))))

\end{lstlisting}

\newpage
\subsection*{7. Напишите функцию, которая добавляет к множеству двухэлементных списков новый
	двухэлементный список, если его там нет.}

\begin{lstlisting}[label=6xd, caption=Решение задания №7, language=lisp]	
(defun is-double-in-set (lst double) 
	(cond ((null lst) nil) 
			((and (eql (caar lst) (car double)) (eql (cadar lst) (cadr double))) t)
			(t (is-double-in-set (cdr lst) double))))
			
(defun inner-append-double-lst (lst double)
	(cond ((null (cdr lst)) (cons (car lst) (cons double nil)))
			(t (cons (car lst) (inner-append-double-lst (cdr lst) double)))))
			
(defun append-double-lst (lst double)
	(if (is-double-in-set lst double)
		lst
		(inner-append-double-lst lst double)))
	
\end{lstlisting}

\subsection*{8. Напишите функцию, которая умножает на заданное число-аргумент первый числовой
	элемент списка из заданного 3-х элементного списка-аргумента, когда}

a) все элементы списка --- числа,

6) элементы списка -- любые объекты.

\begin{lstlisting}[label=6xd, caption=Решение задания №8, language=lisp]
; a)
(defun first-num-mul (lst mul) 
	(cond ((null lst) nil)
			(t (cons (* (car lst) mul) (cdr lst)))))
			
; b)
(defun first-num-mul (lst mul)
	(cond ((null lst) nil)
			((numberp (car lst)) (cons (* (car lst) mul) (cdr lst)))
			((listp (car lst)) (cons (first-num-mul (car lst) mul)) (first-num-mul (cdr lst) mul))
			(t (cons (car lst) (first-num-mul (cdr lst) mul)))))

\end{lstlisting}

\newpage
\subsection*{9. Напишите функцию, select-between, которая из списка-аргумента из 5 чисел выбирает
	только те, которые расположены между двумя указанными границами-аргументами и
	возвращает их в виде списка (упорядоченного по возрастанию списка чисел (+ 2 балла)).}

\begin{lstlisting}[label=6xd, caption=Решение задания №9, language=lisp]
(defun betweenp (el b1 b2)
	(< (* (- el b1) (- el b2)) 0))

(defun select-between (lst b1 b2)
	(cond ((null lst) nil)
			((betweenp (car lst) b1 b2) (cons (car lst) (select-between (cdr lst) b1 b2)))
			(T (select-between (cdr lst) b1 b2))))

\end{lstlisting}

\end{document}