\documentclass[12pt]{report}
\usepackage[utf8]{inputenc}
\usepackage[english, russian]{babel}
\usepackage{listings}
\usepackage{graphicx}
\usepackage{float}
\graphicspath{{imgs/}}
\usepackage{amsmath,amsfonts,amssymb,amsthm,mathtools} 
\usepackage{pgfplots}
\usepackage{filecontents}
\usepackage{indentfirst}
\usepackage{eucal}
\usepackage{enumitem}
\frenchspacing

\usepackage{indentfirst} % Красная строка

\usetikzlibrary{datavisualization}
\usetikzlibrary{datavisualization.formats.functions}

\usepackage{amsmath}
\usepackage{fixltx2e}
\usepackage{caption}


\definecolor{bluekeywords}{rgb}{0,0,1}
\definecolor{greencomments}{rgb}{0,0.5,0}
\definecolor{redstrings}{rgb}{0.64,0.08,0.08}
\definecolor{xmlcomments}{rgb}{0.5,0.5,0.5}
\definecolor{types}{rgb}{0.17,0.57,0.68}

\usepackage{listings}
\lstset{language=[Sharp]C,
	captionpos=t,
	numbers=left, %Nummerierung
	numberstyle=\small, % kleine Zeilennummern
	frame=single, % Oberhalb und unterhalb des Listings ist eine Linie
	stepnumber=1,                   
	numbersep=5pt,                
	showspaces=false,
	tabsize=2,
	showtabs=false,
	breaklines=true,
	showstringspaces=false,
	breakatwhitespace=true,
	escapeinside={(*@}{@*)},
	commentstyle=\color{greencomments},
	morekeywords={partial, var, value, get, set},
	keywordstyle=\color{bluekeywords},
	stringstyle=\color{redstrings},
	basicstyle=\ttfamily\small,
}

\usepackage[left=2cm,right=2cm, top=2cm,bottom=2cm,bindingoffset=0cm]{geometry}
% Для измененных титулов глав:
\usepackage{titlesec, blindtext, color} % подключаем нужные пакеты
\definecolor{gray75}{gray}{0.75} % определяем цвет
\newcommand{\hsp}{\hspace{20pt}} % длина линии в 20pt
% titleformat определяет стиль
\titleformat{\chapter}[hang]{\Huge\bfseries}{\thechapter\hsp\textcolor{gray75}{|}\hsp}{0pt}{\Huge\bfseries}

\usepackage{array}
\newcommand{\head}[2]{\multicolumn{1}{>{\centering\arraybackslash}p{#1}}{#2}}

% plot
\usepackage{pgfplots}
\usepackage{filecontents}
\usetikzlibrary{datavisualization}
\usetikzlibrary{datavisualization.formats.functions}

\begin{document}
%\def\chaptername{} % убирает "Глава"
\thispagestyle{empty}
\begin{titlepage}
	\noindent \begin{minipage}{0.15\textwidth}
		\includegraphics[width=\linewidth]{b_logo}
	\end{minipage}
	\noindent\begin{minipage}{0.9\textwidth}\centering
		\textbf{Министерство науки и высшего образования Российской Федерации}\\
		\textbf{Федеральное государственное бюджетное образовательное учреждение высшего образования}\\
		\textbf{~~~«Московский государственный технический университет имени Н.Э.~Баумана}\\
		\textbf{(национальный исследовательский университет)»}\\
		\textbf{(МГТУ им. Н.Э.~Баумана)}
	\end{minipage}
	
	\noindent\rule{18cm}{3pt}
	\newline\newline
	\noindent ФАКУЛЬТЕТ $\underline{\text{«Информатика и системы управления»}}$ \newline\newline
	\noindent КАФЕДРА $\underline{\text{«Программное обеспечение ЭВМ и информационные технологии»}}$\newline\newline\newline\newline\newline\newline\newline\newline\newline\newline\newline
	
	
	\begin{center}
		\noindent\begin{minipage}{1.3\textwidth}\centering
			\Large\textbf{  Отчет по лабораторной работе №6}\newline
			\textbf{по дисциплине \newline "Функциональное и логическое программирование"}\newline\newline
		\end{minipage}
	\end{center}
	
	\noindent\textbf{Тема} $\underline{\text{Использование функционалов}}$\newline\newline
	\noindent\textbf{Студент} $\underline{\text{Малышев И. А.}}$\newline\newline
	\noindent\textbf{Группа} $\underline{\text{ИУ7-61Б}}$\newline\newline
	\noindent\textbf{Оценка (баллы)} $\underline{\text{~~~~~~~~~~~~~~~~~~~~~~~~~~~}}$\newline\newline
	\noindent\textbf{Преподаватель: } $\underline{\text{Толпинская Н. Б.}}$\newline\newline\newline
	
	\begin{center}
		\vfill
		Москва~---~\the\year
		~г.
	\end{center}
\end{titlepage}


\setcounter{page}{2}

\chapter*{Практические задания}

\subsection*{1. Напишите функцию, которая уменьшает на 10 все числа из списка-аргумента этой
	функции.}

\begin{lstlisting}[label=6xd, caption=Решение задания №1, language=lisp]
(defun lst-minus-10 (lst)
	(mapcar #'(lambda (x) (- x 10)) lst))

\end{lstlisting}

\subsection*{2. Напишите функцию, которая умножает на заданное число-аргумент все числа
	из заданного списка-аргумента, когда}

a) все элементы списка --- числа,

б) элементы списка -- любые объекты.

\begin{lstlisting}[label=6xd, caption=Решение задания №2, language=lisp]
; a)
(defun mult-all-numbers (mult lst)
	(mapcar #'(lambda (el) (* el mult)) lst))
	
; b)
(defun compl-mult-all-numbers (mult lst)
	(mapcar #'(lambda (el) 
				(cond ((listp el) (compl-mult-all-numbers mult el))
						((numberp el) (* el mult))
						(t el)))
			lst))

\end{lstlisting}

\newpage
\subsection*{3. Написать функцию, которая по своему списку-аргументу lst определяет
	является ли он палиндромом (то есть равны ли lst и (reverse lst)).}

\begin{lstlisting}[label=6xd, caption=Решение задания №3, language=lisp]
(defun to-pairs (lst1 lst2)
	(mapcar #'equal lst1 lst2))

(defun is-same (lst1 lst2)
	(every #'identity (to-pairs lst1 lst2)))

(defun is-palyndrome (lst)
	(is-same lst (reverse lst)))

\end{lstlisting}

\subsection*{4. Написать предикат set-equal, который возвращает t, если два его множества-аргумента содержат одни и те же элементы, порядок которых не имеет значения.}

\begin{lstlisting}[label=6xd, caption=Решение задания №4, language=lisp]
(defun comp (lst1 lst2)
	(every #'identity (mapcar #'(lambda (elem)
								(if (find-if #'(lambda (x) (equal elem x)) lst2) T))
						lst1)))

(defun set-equal (lst1 lst2) (if (= (length lst1) (length lst2)) (comp lst1 lst2)))

\end{lstlisting}

\subsection*{5. Написать функцию которая получает как аргумент список чисел, а возвращает список
	квадратов этих чисел в том же порядке.}

\begin{lstlisting}[label=6xd, caption=Решение задания №5, language=lisp]
(defun sqr-lst (lst) 
	(mapcar #'(lambda (x) (* x x)) lst))
	
\end{lstlisting}

\newpage
\subsection*{6.  Напишите функцию, select-between, которая из списка-аргумента, содержащего только
	числа, выбирает только те, которые расположены между двумя указанными границами-аргументами и возвращает их в виде списка (упорядоченного по возрастанию списка чисел
	(+ 2 балла)).}

\begin{lstlisting}[label=6xd, caption=Решение задания №6, language=lisp]
(defun betweenp (el b1 b2)
	(< (* (- el b1) (- el b2)) 0))

(defun select-between (lst b1 b2)
	(mapcan #'(lambda (elem) (if (betweenp elem b1 b2) (list elem))) lst))

\end{lstlisting}

\subsection*{7. Написать функцию, вычисляющую декартово произведение двух своих списков-аргументов. ( Напомним, что А х В это множество всевозможных пар (a b), где а
	принадлежит А, принадлежит В.)}

\begin{lstlisting}[label=6xd, caption=Решение задания №7, language=lisp]	
(defun list-mul (lst1 lst2)
	(mapcan #'(lambda (x)
			(mapcar #'(lambda (y)
						(list x y)) lst2))
									lst1))
	
\end{lstlisting}

\subsection*{8. Почему так реализовано reduce, в чем причина?}

(reduce \#'+ 0) -> 0

(reduce \#'+ ()) -> 0

""\newline
Поведение в данном примере обусловлено работой функции \texttt{+}: при нулевом количестве аргументов возвращает значение 0. Если подать на вход \texttt{reduce} функцию, которая не может обработать 0 аргументов (например,  \texttt{cons}), то вызов \texttt{reduce} с пустым списком в качестве второго аргумента вернет ошибку (\texttt{invalid number of arguments: 0}). При этом, если подано более одного аргумента, то \texttt{reduce} выполняет следующие действия:
\begin{enumerate}
	\item сохраняет первый элемент списка в область памяти (для определенности назовем ее \texttt{acc});
	\item для всех остальных элементов списка выполняет переданную в качестве первого аргумента функцию, подавая на вход 2 аргумента (\texttt{acc} и очередной элемент списка) и сохраняя результат в \texttt{acc}.
\end{enumerate}

\subsection*{9. Пусть list-of-list список, состоящий из списков. Написать функцию, которая вычисляет
	сумму длин всех элементов list-of-list, т.е. например для аргумента ((1 2) (3 4)) -> 4.}

\begin{lstlisting}[label=6xd, caption=Решение задания №9, language=lisp]
(defun sum-lens (list-of-lists)
	(reduce #'(lambda (acc lst) (+ acc (length lst)))
			list-of-lists :initial-value 0))

\end{lstlisting}

\end{document}