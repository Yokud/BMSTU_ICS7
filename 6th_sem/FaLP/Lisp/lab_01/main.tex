\documentclass[12pt]{report}
\usepackage[utf8]{inputenc}
\usepackage[english, russian]{babel}
\usepackage{listings}
\usepackage{graphicx}
\usepackage{float}
\graphicspath{{imgs/}}
\usepackage{amsmath,amsfonts,amssymb,amsthm,mathtools} 
\usepackage{pgfplots}
\usepackage{filecontents}
\usepackage{indentfirst}
\usepackage{eucal}
\usepackage{enumitem}
\frenchspacing

\usepackage{indentfirst} % Красная строка

\usetikzlibrary{datavisualization}
\usetikzlibrary{datavisualization.formats.functions}

\usepackage{amsmath}
\usepackage{fixltx2e}
\usepackage{caption}


\definecolor{bluekeywords}{rgb}{0,0,1}
\definecolor{greencomments}{rgb}{0,0.5,0}
\definecolor{redstrings}{rgb}{0.64,0.08,0.08}
\definecolor{xmlcomments}{rgb}{0.5,0.5,0.5}
\definecolor{types}{rgb}{0.17,0.57,0.68}

\usepackage{listings}
\lstset{language=[Sharp]C,
	captionpos=t,
	numbers=left, %Nummerierung
	numberstyle=\small, % kleine Zeilennummern
	frame=single, % Oberhalb und unterhalb des Listings ist eine Linie
	stepnumber=1,                   
	numbersep=5pt,                
	showspaces=false,
	tabsize=2,
	showtabs=false,
	breaklines=true,
	showstringspaces=false,
	breakatwhitespace=true,
	escapeinside={(*@}{@*)},
	commentstyle=\color{greencomments},
	morekeywords={partial, var, value, get, set},
	keywordstyle=\color{bluekeywords},
	stringstyle=\color{redstrings},
	basicstyle=\ttfamily\small,
}

\usepackage[left=2cm,right=2cm, top=2cm,bottom=2cm,bindingoffset=0cm]{geometry}
% Для измененных титулов глав:
\usepackage{titlesec, blindtext, color} % подключаем нужные пакеты
\definecolor{gray75}{gray}{0.75} % определяем цвет
\newcommand{\hsp}{\hspace{20pt}} % длина линии в 20pt
% titleformat определяет стиль
\titleformat{\chapter}[hang]{\Huge\bfseries}{\thechapter\hsp\textcolor{gray75}{|}\hsp}{0pt}{\Huge\bfseries}

\usepackage{array}
\newcommand{\head}[2]{\multicolumn{1}{>{\centering\arraybackslash}p{#1}}{#2}}

% plot
\usepackage{pgfplots}
\usepackage{filecontents}
\usetikzlibrary{datavisualization}
\usetikzlibrary{datavisualization.formats.functions}

\begin{document}
%\def\chaptername{} % убирает "Глава"
\thispagestyle{empty}
\begin{titlepage}
	\noindent \begin{minipage}{0.15\textwidth}
		\includegraphics[width=\linewidth]{b_logo}
	\end{minipage}
	\noindent\begin{minipage}{0.9\textwidth}\centering
		\textbf{Министерство науки и высшего образования Российской Федерации}\\
		\textbf{Федеральное государственное бюджетное образовательное учреждение высшего образования}\\
		\textbf{~~~«Московский государственный технический университет имени Н.Э.~Баумана}\\
		\textbf{(национальный исследовательский университет)»}\\
		\textbf{(МГТУ им. Н.Э.~Баумана)}
	\end{minipage}
	
	\noindent\rule{18cm}{3pt}
	\newline\newline
	\noindent ФАКУЛЬТЕТ $\underline{\text{«Информатика и системы управления»}}$ \newline\newline
	\noindent КАФЕДРА $\underline{\text{«Программное обеспечение ЭВМ и информационные технологии»}}$\newline\newline\newline\newline\newline\newline\newline\newline\newline\newline\newline
	
	
	\begin{center}
		\noindent\begin{minipage}{1.3\textwidth}\centering
			\Large\textbf{  Отчет по лабораторной работе №1}\newline
			\textbf{по дисциплине \newline "Функциональное и логическое программирование"}\newline\newline
		\end{minipage}
	\end{center}
	
	\noindent\textbf{Тема} $\underline{\text{Списки в Lispe}}$\newline\newline
	\noindent\textbf{Студент} $\underline{\text{Малышев И. А.}}$\newline\newline
	\noindent\textbf{Группа} $\underline{\text{ИУ7-61Б}}$\newline\newline
	\noindent\textbf{Оценка (баллы)} $\underline{\text{~~~~~~~~~~~~~~~~~~~~~~~~~~~}}$\newline\newline
	\noindent\textbf{Преподаватель: } $\underline{\text{Толпинская Н. Б.}}$\newline\newline\newline
	
	\begin{center}
		\vfill
		Москва~---~\the\year
		~г.
	\end{center}
\end{titlepage}


\setcounter{page}{2}
\chapter*{Теоретические вопросы}
""\newline
\textbf{1. Элементы языка: определение, синтаксис, представление в памяти}

Элементами языка Lisp являются атомы и точечные пары (структуры). К атомам относятся:

\begin{itemize}
	\item символы -- набор литер (букв или цифр), начинающихся с буквы.
	\item специальные символы: $\{T, Nil\}$. Используются для обозначения логических констант.
	\item самоопределимые атомы -- натуральные, дробные, вещественные числа, строки (последовательность символов, заключенных в двойные апострофы)
\end{itemize}


Точечные пары ::= (<атом>, <атом>) |

(<атом>, <точечная пара>) |

(<точечная пара>, <атом>) |

(<точечная пара>, <точечная пара>)


""\newline
\indent Список ::= <пустой список> | <непустой список>, где

<пустой список> ::= () | Nil,

<непустой список> ::= (<первый элемент>, <хвост>),

<первый элемент> ::= <S-выражение>,

<хвост> ::= <список>""\newline

\indent \textbf{Список} -- частный случай S-выражения. Любая структура (точечная пара или список) заключаются в круглые скобки:

\begin{itemize}
	\item (A . B) -- точечная пара;
	\item $(A)$ -- список из одного элемента;
	\item $Nil$ или $()$ -- пустой список;
	\item (A . (B . (C . (D ()))))) или (A B C D) -- непустой список;
	\item Элементы списка могут являться списками: $((A)(B)(CD))$
\end{itemize}

Любая непустая структура в Lisp, в памяти представленна списковой ячейкой, хранящей два указателя: на голову и хвост.

Атомы в памяти представляются в виде пяти рядом расположенных указателей: 

\begin{enumerate}
	\item Name -- имя атома;
	\item Value -- значение атома;
	\item Func -- функция, на которую указывает атом;
	\item Property -- свойство;
	\item Package -- пакет.
\end{enumerate}

Атом может указывать одновременно и на значение, и на функцию. В зависимости от контекста воспринимается либо как значение, либо как функция.

""\newline
\textbf{2. Особенности языка Lisp. Структура программы. Символ апостроф.}

Особенности языка:
\begin{itemize}
	\item любая программа, написанная на языке Lisp, может восприниматься и как программа, и как данные;
	\item отдельные функции м. б. значениями в зависимости от контекста;
	\item автоматизированное динамическое распределение памяти.
\end{itemize}

Структура программы представляет собой агрегирование функций. Сама программа представляет собой функцию, некоторые аргументы которой можно воспринимать как функции.

Символ \textquotesingle (апостроф) эквивалентен функции quote – он блокирует вычисление выражения. Таким образом, выражение воспринимается интерпретатором как данные.

""\newline
\textbf{3. Базис языка Lisp. Ядро языка.}

Базис языка представлен:
\begin{itemize}
	\item структурами и атомами;
	\item функциями;
\end{itemize}

Функция -- правило, по которому каждому значению одного или нескольких аргументов ставится в соответствие конкретное значение результата.

Функции, входящие в базис языка: atom, eq, cons, car, cdr, cond, quote, lambda, eval, apply, funcall.

Ядром языка называется базис языка + функции стандартной библиотеки -- это часто используемые функции, созданые на основе базиса языка.

""\newline
\chapter*{Практические задания}

Решение заданий оформлено на листах бумаги, прилагающиеся к отчету.

\end{document}