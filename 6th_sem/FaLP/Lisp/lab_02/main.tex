\documentclass[12pt]{report}
\usepackage[utf8]{inputenc}
\usepackage[english, russian]{babel}
\usepackage{listings}
\usepackage{graphicx}
\usepackage{float}
\graphicspath{{imgs/}}
\usepackage{amsmath,amsfonts,amssymb,amsthm,mathtools} 
\usepackage{pgfplots}
\usepackage{filecontents}
\usepackage{indentfirst}
\usepackage{eucal}
\usepackage{enumitem}
\frenchspacing

\usepackage{indentfirst} % Красная строка

\usetikzlibrary{datavisualization}
\usetikzlibrary{datavisualization.formats.functions}

\usepackage{amsmath}
\usepackage{fixltx2e}
\usepackage{caption}


\definecolor{bluekeywords}{rgb}{0,0,1}
\definecolor{greencomments}{rgb}{0,0.5,0}
\definecolor{redstrings}{rgb}{0.64,0.08,0.08}
\definecolor{xmlcomments}{rgb}{0.5,0.5,0.5}
\definecolor{types}{rgb}{0.17,0.57,0.68}

\usepackage{listings}
\lstset{language=[Sharp]C,
	captionpos=t,
	numbers=left, %Nummerierung
	numberstyle=\small, % kleine Zeilennummern
	frame=single, % Oberhalb und unterhalb des Listings ist eine Linie
	stepnumber=1,                   
	numbersep=5pt,                
	showspaces=false,
	tabsize=2,
	showtabs=false,
	breaklines=true,
	showstringspaces=false,
	breakatwhitespace=true,
	escapeinside={(*@}{@*)},
	commentstyle=\color{greencomments},
	morekeywords={partial, var, value, get, set},
	keywordstyle=\color{bluekeywords},
	stringstyle=\color{redstrings},
	basicstyle=\ttfamily\small,
}

\usepackage[left=2cm,right=2cm, top=2cm,bottom=2cm,bindingoffset=0cm]{geometry}
% Для измененных титулов глав:
\usepackage{titlesec, blindtext, color} % подключаем нужные пакеты
\definecolor{gray75}{gray}{0.75} % определяем цвет
\newcommand{\hsp}{\hspace{20pt}} % длина линии в 20pt
% titleformat определяет стиль
\titleformat{\chapter}[hang]{\Huge\bfseries}{\thechapter\hsp\textcolor{gray75}{|}\hsp}{0pt}{\Huge\bfseries}

\usepackage{array}
\newcommand{\head}[2]{\multicolumn{1}{>{\centering\arraybackslash}p{#1}}{#2}}

% plot
\usepackage{pgfplots}
\usepackage{filecontents}
\usetikzlibrary{datavisualization}
\usetikzlibrary{datavisualization.formats.functions}

\begin{document}
%\def\chaptername{} % убирает "Глава"
\thispagestyle{empty}
\begin{titlepage}
	\noindent \begin{minipage}{0.15\textwidth}
		\includegraphics[width=\linewidth]{b_logo}
	\end{minipage}
	\noindent\begin{minipage}{0.9\textwidth}\centering
		\textbf{Министерство науки и высшего образования Российской Федерации}\\
		\textbf{Федеральное государственное бюджетное образовательное учреждение высшего образования}\\
		\textbf{~~~«Московский государственный технический университет имени Н.Э.~Баумана}\\
		\textbf{(национальный исследовательский университет)»}\\
		\textbf{(МГТУ им. Н.Э.~Баумана)}
	\end{minipage}
	
	\noindent\rule{18cm}{3pt}
	\newline\newline
	\noindent ФАКУЛЬТЕТ $\underline{\text{«Информатика и системы управления»}}$ \newline\newline
	\noindent КАФЕДРА $\underline{\text{«Программное обеспечение ЭВМ и информационные технологии»}}$\newline\newline\newline\newline\newline\newline\newline\newline\newline\newline\newline
	
	
	\begin{center}
		\noindent\begin{minipage}{1.3\textwidth}\centering
			\Large\textbf{  Отчет по лабораторной работе №2}\newline
			\textbf{по дисциплине \newline "Функциональное и логическое программирование"}\newline\newline
		\end{minipage}
	\end{center}
	
	\noindent\textbf{Тема} $\underline{\text{Определение функций пользователя}}$\newline\newline
	\noindent\textbf{Студент} $\underline{\text{Малышев И. А.}}$\newline\newline
	\noindent\textbf{Группа} $\underline{\text{ИУ7-61Б}}$\newline\newline
	\noindent\textbf{Оценка (баллы)} $\underline{\text{~~~~~~~~~~~~~~~~~~~~~~~~~~~}}$\newline\newline
	\noindent\textbf{Преподаватель: } $\underline{\text{Толпинская Н. Б.}}$\newline\newline\newline
	
	\begin{center}
		\vfill
		Москва~---~\the\year
		~г.
	\end{center}
\end{titlepage}


\setcounter{page}{2}
\chapter*{Теоретические вопросы}

\section*{1. Базис Lisp}

Базис языка представлен:
\begin{itemize}
	\item структурами и атомами;
	\item функциями;
\end{itemize}

Функция -- правило, по которому каждому значению одного или нескольких аргументов ставится в соответствие конкретное значение результата.

Функции, входящие в базис языка: atom, eq, cons, car, cdr, cond, quote, lambda, eval, apply, funcall.


\section*{2. Классификация функций}

\begin{itemize}
	\item "чистая" функция -- соответствует математической функции;
	\item специальные функции или формы -- могут принимать несколько аргументов;
	\item псевдофункции -- создают «эффект» -- отображение на экране процесса обработки данных и т.п.;
	\item функции с вариантами значения;
	\item функционалы -- принимают в качестве аргумента функцию или возвращают функцию;
	\item базисные функции (cons, car, cdr, atom, cond, eq, quote, eval, lambda, apply, funcall).
\end{itemize}

\section*{3. Способы создание функций}

Определение функций пользователя в Lisp-е возможно двумя способами:
\begin{itemize}
	\item с использованием Лямбда-нотаци (функции без имени): (lambda (<аргументы>) (<тело>));
	\item с использованием макро определения DEFUN: (defun <имя> (<аргументы>) (<тело>)).
\end{itemize}

\section*{4. Функции Car и Cdr}

Функции $car$, $cdr$ являются базовыми функциями доступа к данным. car принимает точечную пару или список в качестве аргумента и возвращает первый элемент или $Nil$, соответственно. $cdr$ принимает точечную пару или список в качестве аргумента и возвращает все элементы кроме первого или $Nil$, соответственно.

\section*{5. Назначение и отличие в работе Cons и List}

Функции $list$, $cons$ являются функциями создания списков ($cons$ – базовая, $list$ – нет). $cons$ создает списковую ячейку и устанавливает два указателя на аргументы. $list$ принимает переменное число аргументов и возвращает список, элементы которого – переданные в функцию аргументы.

Отличие в работе $cons$ и $list$ заключается в том, что $cons$ создаёт одну списковую ячейку и устанавливает два указателя на аргументы, а $list$ создаёт списковые ячейки для каждого аргумета, где голова -- сам элемент списка, хвост -- остальная часть списка.

\chapter*{Практические задания}

\subsection*{1. Составить диаграмму вычисления следующих выражений:}

(equal 3 (abs - 3))  

(equal (* 2 3) (+ 7 2))

(equal (+ 1 2) 3) 

(equal (- 7 3) (* 3 2))

(equal (* 4 7) 21) 

(equal (abs (- 2 4)) 3))

""\newline
Диаграмма вычисления оформлена на листе бумаги, прилагающийся к отчету.

\subsection*{2. Написать функцию, вычисляющую гипотенузу прямоугольного треугольника по заданным катетам и составить диаграмму её вычисления.}

\begin{lstlisting}[label=6xd, caption=Решение задания №2, language=lisp]
	(defun hypot (a b) (sqrt (+ (* a a) (* b b))))
\end{lstlisting}
	
Диаграмма вычисления оформлена на листе бумаги, прилагающийся к отчету.

\subsection*{3. Написать функцию, вычисляющую объем параллелепипеда по 3-м его сторонам, и составить диаграмму ее вычисления.}

\begin{lstlisting}[label=6xd, caption=Решение задания №3, language=lisp]
	(defun volume (a b c) (* a b c))
\end{lstlisting}

Диаграмма вычисления оформлена на листе бумаги, прилагающийся к отчету.

\subsection*{4. Каковы результаты вычисления следующих выражений?(объяснить возможную ошибку и варианты ее устранения)}

(list 'a c) -> the variable c is unbound; (list 'a 'c) -> (a c)

(cons 'a 'b 'c) -> invalid number of arguments; (cons 'a 'b) -> (a . b)

(cons 'a (b c)) -> undefined function b; (cons 'a '(b c)) -> (a b c)

(list 'a (b c)) -> undefined function b; (list 'a '(b c)) -> (a (b c))

(cons 'a '(b c)) -> (a b c)

(list a '(b c)) -> the variable a is unbound; (list 'a '(b c)) -> (a (b c))

(caddr (1 2 3 4 5)) -> 3

(list (+ 1 '(length '(1 2 3)))) -> (LENGTH '(1 2 3)) is not of type NUMBER; (list (+ 1 (length '(1 2 3)))) -> 4


\subsection*{5. Написать функцию longer\_than от двух списков-аргументов, которая возвращает Т, если первый аргумент имеет большую длину.}

\begin{lstlisting}[label=5xd,caption=Решение задания №5, language=lisp]
	(defun longer_than (l1 l2) (> (length l1) (length l2)))
\end{lstlisting}

\subsection*{6. Каковы результаты вычисления следующих выражений?}

(cons 3 (list 5 6)) -> (3 5 6)

(cons 3 '(list 5 6)) -> (3 list 5 6)

(list 3 'from 9 'lives (- 9 3)) -> (3 from 9 lives 6)

(+ (length for 2 too)) (car '(21 22 23))) -> illegal function call; (+ (length '(for 2 too)) (car '(21 22 23))) -> 24

(cdr '(cons is short for ans)) -> (is short for ans)

(car (list one two)) -> the variable one is unbound; (car (list 'one 'two)) -> one

(car (list 'one 'two)) -> one

\subsection*{7. Дана функция (defun mystery (x) (list (second x) (first x))). Какие результаты вычисления следующих выражений? }

(mystery (one two)) -> undefined function one; (mystery '(one two)) -> (two one)

(mystery one 'two) -> invalid number of arguments; (mystery '(one two)) -> (two one)

(mystery (last one two)) -> undefined function last; (mystery '(last one two)) -> (one last)

(mystery free) -> unbound variable free; (mystery '(free)) -> (NIL free)

\subsection*{8. Написать функцию, которая переводит температуру в системе Фаренгейта температуру по Цельсию (defum f-to-c (temp)…).}

Формулы: c = 5/9*(f-32.0); f= 9/5*c+32.0.

Как бы назывался роман Р.Брэдбери "+451 по Фаренгейту" в системе по Цельсию?

\begin{lstlisting}[label=5xd,caption=Решение задания №8, language=lisp]
	(defun f-to-c (temp) (* (/ 5 9) (- temp 32.0)))
\end{lstlisting}

Роман Р.Брэдбери "+451 по Фаренгейту" в системе по Цельсию назывался бы "+232.778 по Цельсию".

\subsection*{9. Что получится при вычисления каждого из выражений?}

(list 'cons t NIL) -> (cons T NIL)

(eval (list 'cons t NIL)) -> (T)

(eval (eval (list 'cons t NIL))) -> The function T is undefined

\subsection*{Дополнительно}

\begin{enumerate}
	\item  Написать функцию, вычисляющую катет по заданной гипотенузе и другому катету
	прямоугольного треугольника, и составить диаграмму ее вычисления.
	
	\begin{lstlisting}[label=third,caption=Решение задания, language=lisp]
	(defun get_cathetus (c a) (sqrt (- (* c c) (* a a))))
	\end{lstlisting}

	Диаграмма вычисления оформлена на листе бумаги, прилагающийся к отчету.
	
	\item Написать функцию, вычисляющую площадь трапеции по ее основаниям и
	высоте, и составить диаграмму ее вычисления.
	
	\begin{lstlisting}[label=4xxd,caption=Решение задания, language=lisp]
	(defun trapezium_area (a b h) (* (/ (+ a b) 2) h))
	\end{lstlisting}
	
	Диаграмма вычисления оформлена на листе бумаги, прилагающийся к отчету.
\end{enumerate}

\end{document}