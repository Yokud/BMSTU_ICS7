\chapter{Технологический раздел}

\section{Выбор языка и среды программирования}

Модуль ядра написан на языке программирования C \cite{c-language}. Выбор языка основан на том, что исходный код системы, все модули ядра и драйверы операционной системы Linux написаны на языке C. 

В качестве компилятора выбран gcc \cite{gcc}.

В качестве среды программирования выбрана среда Visual Studio Code \cite{vscode}. 

\section{Перехват функций}

В листинге 3.1 представлена структура struct ftrace\_hook, которая описывает каждую перехватываемую функцию.\newline

\includelisting{ftrace_hook.c}{Структура перехватываемой функции}

""\newline\indent Пользователю необходимо заполнить только первые три поля: name, function, original. Остальные поля считаются деталью реализации. Описание всех перехватываемых функций можно собрать в массив и использовать макросы, чтобы повысить компактность кода. В листинге 3.2 представлен массив перехватываемых функций.\newline

\newpage

\includelisting{ftrace_hook_arr.c}{Массив перехватываемых функций}

\section{Инициализация ftrace}

Для инициализации необходимо найти и сохранить адрес функции, которую будет перехватывать разрабатываемый модуль ядра. ftrace позволяет трассировать функции по имени, но при этом все равно надо знать адрес оригинальной функции, чтобы вызывать ее.

Найти адрес можно с помощью kallsyms -- списка всех символов в ядре. В этот список входят все символы, не только экспортируемые для модулей. Получение адреса перехватываемой функции представлено в листинге 3.3.\newline

\includelisting{resolve_hook_adress.c}{Получение адреса перехватываемой функции}

""\newline\indent В листинге 3.4 инициализируется структура ftrace\_ops. В ней обязательным полем является лишь func, указывающая на коллбек, но также необходимо установить некоторые важные флаги.\newline

\includelisting{install_hook.c}{Регистрация перехвата}

\newpage

""\newline\indent В листинге 3.5 представлена функция дерегистрации перехвата.

\includelisting{remove_hook.c}{Дерегистрация перехвата}

\section{Функции-обертки для перехватываемых функций}

Для перехвата необходимо определить функции-обертки. Очень важно в точности соблюдать сигнатуру функции: порядок и типы аргументов и возвращаемого значения. Оригинальные функции были взяты из исходных кодов ядра Linux \cite{linux_core}.

На листингах 3.6-3.9 представлены реализации функций-оберток для системных вызовов и считывания информации с блочного устройства.
\newpage

\includelisting{sys_clone.c}{Функция-обертка для sys\_clone}

\includelisting{sys_execve.c}{Функция-обертка для sys\_execve}

\newpage

\includelisting{read_page.c}{Функция-обертка для bdev\_read\_page}

\includelisting{write_page.c}{Функция-обертка для bdev\_write\_page}
	
\section{Сбор данных для визуализации}

Собранные данные о вызове функций хранятся в лог-файле /var/log/syslog. Для того, чтобы передать собранные метрики в платформу Grafana для визуализации, необходимо настроить Promtail для считывания данных из лог-файла и их дальнейшую передачу в Loki для хранения. В листинге 3.10 представлен фрагмент конфигурационного файла Promtail.\newline

\includelisting{trace_config.c}{Сбор метрик /var/log/syslog}

""\newline\indent Собранные данные Loki отправляет на порт 3100. На рисунке \ref{img:grafana} представлена настройка Grafana для прослушивания Loki на порту 3100.

\includeimage
{grafana}
{f}
{H}
{\textwidth}
{Настройка Grafana}	

\section{Выводы}

В данном разделе был обоснован выбор языка программирования и средств программирования, представлены листинги реализованных функций.