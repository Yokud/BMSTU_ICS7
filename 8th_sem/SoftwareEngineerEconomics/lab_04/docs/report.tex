\documentclass[12pt]{report}
\usepackage[utf8]{inputenc}
\usepackage[english, russian]{babel}
\usepackage{listings}
\usepackage{graphicx}
\usepackage{float}
\graphicspath{{imgs/}}
\usepackage{amsmath,amsfonts,amssymb,amsthm,mathtools} 
\usepackage{pgfplots}
\usepackage{filecontents}
\usepackage{indentfirst}
\usepackage{eucal}
\usepackage{enumitem}
\frenchspacing

\usepackage{indentfirst} % Красная строка

\usetikzlibrary{datavisualization}
\usetikzlibrary{datavisualization.formats.functions}

\usepackage{amsmath}
\usepackage{fixltx2e}
\usepackage{caption}


\definecolor{bluekeywords}{rgb}{0,0,1}
\definecolor{greencomments}{rgb}{0,0.5,0}
\definecolor{redstrings}{rgb}{0.64,0.08,0.08}
\definecolor{xmlcomments}{rgb}{0.5,0.5,0.5}
\definecolor{types}{rgb}{0.17,0.57,0.68}

\usepackage{listings}
\lstset{language=[Sharp]C,
	captionpos=t,
	numbers=left, %Nummerierung
	numberstyle=\small, % kleine Zeilennummern
	frame=single, % Oberhalb und unterhalb des Listings ist eine Linie
	stepnumber=1,                   
	numbersep=5pt,                
	showspaces=false,
	tabsize=2,
	showtabs=false,
	breaklines=true,
	showstringspaces=false,
	breakatwhitespace=true,
	escapeinside={(*@}{@*)},
	commentstyle=\color{greencomments},
	morekeywords={partial, var, value, get, set},
	keywordstyle=\color{bluekeywords},
	stringstyle=\color{redstrings},
	basicstyle=\ttfamily\small,
}

\usepackage[left=2cm,right=2cm,top=1cm,bottom=2cm,bindingoffset=0cm]{geometry}
% Для измененных титулов глав:
\usepackage{titlesec, blindtext, color} % подключаем нужные пакеты
\definecolor{gray75}{gray}{0.75} % определяем цвет
\newcommand{\hsp}{\hspace{20pt}} % длина линии в 20pt
% titleformat определяет стиль
\titleformat{\chapter}[hang]{\Huge\bfseries}{\thechapter\hsp\textcolor{gray75}{|}\hsp}{0pt}{\Huge\bfseries}

\usepackage{array}
\newcommand{\head}[2]{\multicolumn{1}{>{\centering\arraybackslash}p{#1}}{#2}}

% plot
\usepackage{pgfplots}
\usepackage{filecontents}
\usetikzlibrary{datavisualization}
\usetikzlibrary{datavisualization.formats.functions}

\begin{document}
	%\def\chaptername{} % убирает "Глава"
	\thispagestyle{empty}
	\begin{titlepage}
		\noindent \begin{minipage}{0.15\textwidth}
			\includegraphics[width=\linewidth]{b_logo}
		\end{minipage}
		\noindent\begin{minipage}{0.9\textwidth}\centering
			\textbf{Министерство науки и высшего образования Российской Федерации}\\
			\textbf{Федеральное государственное бюджетное образовательное учреждение высшего образования}\\
			\textbf{~~~«Московский государственный технический университет имени Н.Э.~Баумана}\\
			\textbf{(национальный исследовательский университет)»}\\
			\textbf{(МГТУ им. Н.Э.~Баумана)}
		\end{minipage}
		
		\noindent\rule{18cm}{3pt}
		\newline\newline
		\noindent ФАКУЛЬТЕТ $\underline{~~~~~~~~~~~~~~~~~~~\text{«Информатика, искусственный интеллект и системы управления»}~~~~~~~~~~~~~~~~~~~~~~~~~~~~~~~~~~~~~}$ \newline\newline
		\noindent КАФЕДРА $\underline{~~~~~~~~~~~~~\text{«Программное обеспечение ЭВМ и информационные технологии»}~~~~~~~~~~~~~~~~~~~~~~~}$\newline\newline\newline\newline\newline\newline\newline\newline\newline
		
		
		\begin{center}
			\noindent\begin{minipage}{1.3\textwidth}\centering
				\Large\textbf{  Отчет по лабораторной работе №4}\newline
				\textbf{по дисциплине \newline "Экономика программной инженерии"}\newline\newline
			\end{minipage}
		\end{center}
		
		\noindent\textbf{Тема} $\underline{\text{Актуализация параметров проекта. Ввод фактических данных для задач и просмотр}}$\newline\newline
		\indent\indent $\underline{\text{отклонений от контрольного плана}}$\newline\newline
		\noindent\textbf{Студент} $\underline{\text{Малышев И. А.}}$\newline\newline
		\noindent\textbf{Группа} $\underline{\text{ИУ7-81Б}}$\newline\newline
		\noindent\textbf{Оценка (баллы)} $\underline{\text{~~~~~~~~~~~~~~~~~~~~~~~~~~~}}$\newline\newline
		\noindent\textbf{Преподаватель: } $\underline{\text{Барышникова М. Ю.}}$\newline\newline\newline
		
		\begin{center}
			\vfill
			Москва~---~\the\year
			~г.
		\end{center}
	\end{titlepage}
	
	\setcounter{page}{2}
	
	\chapter{Лабораторная работа}

\section*{Цель работы}

Лабораторная работа №4 выполняется на основе лабораторной работы №3.
Ее цель -- знакомство с возможностями программы Microsoft Project по
контролю за ходом реализации проекта.

\section*{Содержание проекта}

Команда разработчиков из 16 человек занимается созданием карты города на основе собственного модуля отображения. Проект должен быть завершен в течение 6 месяцев. Бюджет проекта: 50 000 рублей.

\section*{Условия}

\begin{enumerate}
	\item Дата отчета -- 6 мая.
	\item С 13 марта 2 недели болел 3D-аниматор. Его работу в этот период на 20\% выполнял художник-дизайнер, сократив работу по своим задачам до 80\%. За совмещение работ в начале апреля художнику–дизайнеру была выплачена премия в размере 500 рублей.
	\item Задача №7 завершилась на 5 рабочих дней позже.
	\item Фактические трудозатраты на выполнение задачи №15 оказались на 20\% больше.
	\item Задача №17 выполнена на 40\%. С 12 апреля на 10\% повысилась зарплата мультимедиа корреспондента.
	\item С 1 апреля отказались от аренды сервера и купили собственный за 3200 рублей.
	\item С 3 апреля в течение двух недель с 10 до 14 часов организуется повышение квалификации для Программистов №1-4. После завершения обучения их зарплата будет увеличена на 10\%. Для проведения обучения программистов нанят преподаватель с оплатой 700 рублей в неделю
\end{enumerate}

\section*{Дата отчета}

Зададим дату отчета 6 мая:

\begin{figure}[H]
	\begin{center}
		\includegraphics[width=0.7\textwidth]{imgs/task_1_0.png}
	\end{center}
\end{figure}

\section*{Фактические данные}

Внесем фактические данные для отдельных задач проекта.

С 13 марта 2 недели болел 3D-аниматор

\begin{figure}[H]
	\begin{center}
		\includegraphics[width=0.7\textwidth]{imgs/task_1_1.png}
	\end{center}
\end{figure}

Его работу в этот период на 20\% выполнял художник-дизайнер, сократив работу по своим задачам до 80\%

\begin{figure}[H]
	\begin{center}
		\includegraphics[width=0.7\textwidth]{imgs/task_1_2.png}
	\end{center}
\end{figure}

\begin{figure}[H]
	\begin{center}
		\includegraphics[width=0.7\textwidth]{imgs/task_1_3.png}
	\end{center}
\end{figure}

За совмещение работ в начале апреля художнику–дизайнеру была выплачена премия в размере 500 рублей

\begin{figure}[H]
	\begin{center}
		\includegraphics[width=\textwidth]{imgs/task_1_4.png}
	\end{center}
\end{figure}

Задача №7 завершилась на 5 рабочих дней позже

\begin{figure}[H]
	\begin{center}
		\includegraphics[width=0.7\textwidth]{imgs/task_1_5.png}
	\end{center}
\end{figure}

Фактические трудозатраты на выполнение задачи №15 оказались на 20\% больше

\begin{figure}[H]
	\begin{center}
		\includegraphics[width=\textwidth]{imgs/task_1_6.png}
	\end{center}
\end{figure}

Задача №17 выполнена на 40\%

\begin{figure}[H]
	\begin{center}
		\includegraphics[width=0.7\textwidth]{imgs/task_1_7.png}
	\end{center}
\end{figure}

С 12 апреля на 10\% повысилась зарплата мультимедиа корреспондента

\begin{figure}[H]
	\begin{center}
		\includegraphics[width=0.85\textwidth]{imgs/task_1_8.png}
	\end{center}
\end{figure}

С 1 апреля отказались от аренды сервера и купили собственный за 3200 рублей

\begin{figure}[H]
	\begin{center}
		\includegraphics[width=0.85\textwidth]{imgs/task_1_9.png}
	\end{center}
\end{figure}

\begin{figure}[H]
	\begin{center}
		\includegraphics[width=\textwidth]{imgs/task_1_10.png}
	\end{center}
\end{figure}

С 3 апреля в течение двух недель с 10 до 14 часов организуется повышение квалификации для Программистов №1-4

\begin{figure}[H]
	\begin{center}
		\includegraphics[width=0.85\textwidth]{imgs/task_1_11.png}
	\end{center}
\end{figure}

\begin{figure}[H]
	\begin{center}
		\includegraphics[width=0.85\textwidth]{imgs/task_1_12.png}
	\end{center}
\end{figure}

После завершения обучения их зарплата будет увеличена на 10\%

\begin{figure}[H]
	\begin{center}
		\includegraphics[width=0.85\textwidth]{imgs/task_1_13.png}
	\end{center}
\end{figure}

Для проведения обучения программистов нанят преподаватель с оплатой 700 рублей в неделю

\begin{figure}[H]
	\begin{center}
		\includegraphics[width=\textwidth]{imgs/task_1_14.png}
	\end{center}
\end{figure}

\begin{figure}[H]
	\begin{center}
		\includegraphics[width=0.7\textwidth]{imgs/task_1_15.png}
	\end{center}
\end{figure}

После внесения фактических данных выяснилось, что проект вышел за рамки бюджета

\begin{figure}[H]
	\begin{center}
		\includegraphics[width=\textwidth]{imgs/task_1_16.png}
	\end{center}
\end{figure}

Также появилась перегрузка ресурсов

\begin{figure}[H]
	\begin{center}
		\includegraphics[width=\textwidth]{imgs/task_1_17.png}
	\end{center}
\end{figure}

Расписание

\begin{figure}[H]
	\begin{center}
		\includegraphics[width=\textwidth]{imgs/task_1_18.png}
	\end{center}
\end{figure}

\section*{Выравнивание}

Уберем у 3 и 4 совещаний заболевшего 3D аниматора:

\begin{figure}[H]
	\begin{center}
		\includegraphics[width=\textwidth]{imgs/task_1_19.png}
	\end{center}
\end{figure}

Произведем выравнивание

\begin{figure}[H]
	\begin{center}
		\includegraphics[width=\textwidth]{imgs/task_1_20.png}
	\end{center}
\end{figure}

\section*{Сравнение показателей}

Сроки:

\begin{itemize}
	\item Плановое окончание: 23.08.23
	\item Текущее окончание: 28.08.23
	\item Максимальное окончание: 1.09.23
\end{itemize}

Затраты:

\begin{itemize}
	\item Плановые затраты: 48 766 рублей
	\item Текущие затраты: 54 076 рублей
	\item Бюджет: 50 000 рублей
\end{itemize}

Вывод: перегрузку ресурсов в данных условиях уджалось устранить выравниванием. Срок проекта сдвинулся на 5 дней, что еще допустимо в рамках плана. Также увеличились затраты настолько, что превышают бюджет на 4 076 рублей. Требуется корректировка ресурсов.

\section*{Линия прогресса}

Отобразим линию прогресса проекта

\begin{figure}[H]
	\begin{center}
		\includegraphics[width=\textwidth]{imgs/task_1_21.png}
	\end{center}
\end{figure}

\section*{Устранение отклонений}

Создание рабочей версии ядра лежит на критическом пути и занимает много времени, ею занимаются все программисты

\begin{figure}[H]
	\begin{center}
		\includegraphics[width=\textwidth]{imgs/task_1_22.png}
	\end{center}
\end{figure}

Добавим программиста-стажера

\begin{figure}[H]
	\begin{center}
		\includegraphics[width=\textwidth]{imgs/task_1_23.png}
	\end{center}
\end{figure}

И добавим его в две задачи

\begin{figure}[H]
	\begin{center}
		\includegraphics[width=\textwidth]{imgs/task_1_24.png}
	\end{center}
\end{figure}

\begin{figure}[H]
	\begin{center}
		\includegraphics[width=\textwidth]{imgs/task_1_25.png}
	\end{center}
\end{figure}

После чего получим дату окончания 1.08.23 (даже раньше, чем в базовом) и затраты 49 888 рублей

\begin{figure}[H]
	\begin{center}
		\includegraphics[width=\textwidth]{imgs/task_1_26.png}
	\end{center}
\end{figure}
	
	\chapter{Выводы}

В данной работе программисты требуют высокую плату (особенно ведущий), поэтому там, где возможно использовать обычных программистов, то следует их загрузить работой, тем самым освободив ведущего от большей части работы сохранив бюджет.

Для введения периодической задачи следует предусмотреть подключение к ней ресурсов с особенной нормой оплаты. Особенная плата -- 48766 р., обычная -- 63008 р., разница -- 14242 р.

Решение вопроса длительного проекта можно решить оптимизацией ресурсов: увеличив ресурсы мы решаем задачи быстрее, соответственно уменьшая длительность проекта. В результате получим 48766 рублей затрат на проект длительностью 24.61 недель. При этом отношение <<Трудозатраты -- затраты>> сильно не изменилось.
	
\end{document}