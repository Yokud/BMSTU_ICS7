\chapter{Выводы}

В результате выполнения лабораторной работы был разработан программный инструмент для оценки проекта по методике COCOMO. Были изучены существующие методики предварительной оценки параметров программного проекта, а также проведена практическая оценка затрат проекта.

По результатам применения методики оценки COCOMO можно заключить, что она пригодна для общей предварительной оценки всего проекта и позволяет получить приблизительные значения трудозатрат и времени на реализацию проекта, разделенные на стадии его жизненного цикла. Однако для постоянного отслеживания состояния проекта рекомендуется использовать другие методики управления проектами с использованием различных программных средств, которые позволяют актуализировать данные проекта в реальном времени и своевременно адаптироваться к непредвиденным изменениям в проекте.