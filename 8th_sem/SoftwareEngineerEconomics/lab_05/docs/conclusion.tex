\chapter{Выводы}

В данной работе был произведен анализ затрат по методике освоенного объема, из которого выяснилось, что проект по состоянию на дату отчета успевает в срок и при этом не выходит за пределы сметы и не имеется перерасход средств. Также из отчетов было выявлено, что руководитель проекта будет испытывать наибольшую потребность в деньгах на 19 неделе и превышение бюджетной стоимости некоторых задач связаны либо с повышением зарплат работников, либо с болезнью одного из сотрудников, из-за чего другому приходилось частично <<работать за двоих>>.

Также был предложен вариант декомпозиции задач, в котором разработка документации и веб-сайта выполняются параллельно разработке основного ПО, а не начинались только после его создания. Это позволило сократить срок разработки на 10 дней и расходы на 672 рубля, но при этом этот вариант декомпозиции неидеален: из диаграммы Ганта можно увидеть простои некоторых задач, т. е. они не выполняются непрерывно.

Из двух вариантов декомпозиции я бы предпочел свой, а не исходный: хоть он и может вызвать головную боль менеджеров с планированием задач и переключением между ними, но он позволяет распараллелить задачи по профессиям, а также сократить срок разработки на 10 дней и расходы на 672 рубля.